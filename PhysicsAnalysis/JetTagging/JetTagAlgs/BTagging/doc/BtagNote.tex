\documentclass[a4paper,12pt]{article}
\usepackage{epsfig}
\usepackage{geometry}
\geometry{verbose,a4paper,tmargin=2.9cm,bmargin=2.3cm,lmargin=2.5cm,rmargin=2.5cm}
\usepackage[english]{babel}
\input newcommands.tex
%\usepackage{rotate}

%opening
\title{B-Tagging EDM and Implementation}
%\author{F. {\AA}kesson, A. Wildauer}
% 
\special{!userdict begin /bop-hook{gsave 130 130 translate
90 rotate /Palatino-Bold findfont 25 scalefont setfont 100 90 moveto 0.7 setgray 
  (03.12.2004 - B-Tagging Draft 0.3) show grestore}def end}

% -------------------------------------------------------------------
\author{
  F. {\AA}kesson, CERN, Geneva, Switzerland\\
  A. Wildauer, Leopold-Franzens-University, Innsbruck, Austria
}

\begin{document}

\maketitle

\begin{abstract}
For various physics analyses in ATLAS, reliable and efficient b-tagging
algorithms are important. In the harsh enviroment of the LHC ($\sim$23
inelastic collisions every 25 ns) this task turns out to be particularily 
challenging. One of the guiding principles in developing the b-tagging 
packages is a strong focus on modularity and defined interfaces using the 
advantages of object oriented C++. The benefit is the easy expandability 
of the b-tagging with additional tag strategies integrated in the Athena 
framework. Various implementations of tagging tools and strategies 
dedicated to b-tagging using the full reconstruction of simulated ATLAS 
events are presented.
\end{abstract}

\newpage

\section{Introduction}\label{intro}
This section will describe the need and necessity of b-tagging algorithms
for the future physics program of ATLAS.

\section{Theoretical Background}\label{theorie}
Due to the relatively long life-time of the b-flavoured mesons the decay 
will on average take place a few mm from the primary vertex. The signature 
in the detector will be a secondary vertex which can be reconstructed with 
the tools presented here. In addition to the secondary vertex finding other 
characteristics of jets, such as jet-kinematics, semi-leptonic decays of 
the B-mesons, and impact parameters can be used to separate b-flavoured jets 
from jets with lighter flavours. The resulting b-tagging objects will be 
part of the AOD production and will support physicists searching for 
processes with final state b-jets.

\begin{itemize}
\item Lifetime information: signed impact parameters, secondary vertexing, etc.
\item Leptonic decays: Even if it is a small fraction ($\approx 10\%$) of the
b-mesons decaying leptonically the leptons have characteristic properties.
\item Kinematic properties of the b-jet.
\end{itemize}

\section{Common Interfaces and Data Flow}\label{interfaces}
The basic information flow of the b-tagging is shown in 
fig.~\ref{fig:infoflow}. The {\tt BJetBuilder} algorithm 
(sec.~\ref{fig:bjetalg}) reads the reconstructed primary vertex and jets 
from the transient data storage, creates a BJet (sec.~\ref{sec:bjet}) for
each jet to be tagged and
passes it on for tagging to the different Athena tagging tools.
The tools creates a tag specific {\tt BInfo}-object which is added to the
{\tt BJet}. The collection of BJets from the event are added to the 
AOD-stream. The information can also be written out into a ntuple file.

\begin{figure}[htbp]
  \centerline{\epsfig{file=images/InfoFlow.eps,width=12.0cm}}
  \caption[]{\label{fig:infoflow}\sl
    The schematic information flow for the b-tagger. The BJetBuilder algorithm
    reads the reconstructed primary vertex and jets from the transient data 
    storage, creates a BJet which is passed on for tagging to the different 
    Athena tagging tools. The collection of BJets from the event are added to
    the AOD-stream. The information can also be written out into a ntuple file.}
\end{figure}

\subsection{The {\tt BJetBuilder} Algorithm}\label{sec:bjetalg}
The b-tagging top algorithm has the following tasks:
\begin{itemize}
  \item Read the jets and the primary vertex from the transient data storage
  \item Perform basic preselection on the tracks in the jet
  \item Call the different b-tagging tools
  \item Combine the results of the b-tagging tools
  \item Write the resulting BJets to the transient data storage
\end{itemize}

\begin{figure}[htbp]
  \centerline{\epsfig{file=images/BTagBasic.eps,width=5.0cm}}
  \caption[]{\label{fig:btagbasic}\sl
    The scematic implementation of the reconstructed vertex class.}
\end{figure}

\subsection{The {\tt BJet}-Object}\label{sec:bjet}
The BJet class is the container for all b-tagging related properties 
of the jet. To ensure common interfaces with other AOD particle objects the
BJet inherits from {\tt ParticleBase} and {\tt P4PxPyPzE} 
(fig.~\ref{fig:bjet}a). The {\tt ParticleBase} gives the particle basic 
attributes, e.g. charge, originating vertex and if the jet is reconstructed 
from simulated data, while the {\tt P4PxPyPzE} inheritance gives the BJet 
the basic kinematic attributes, e.g. $p_t$ and Energy. The BJet is a 
navigable of {\tt Jet}.

\begin{figure}[htbp]
  \centerline{
    \epsfig{file=images/BJet.eps,width=6.0cm}
    \epsfig{file=images/BInfoBasic.eps,width=6.0cm}}
  \caption[]{\label{fig:bjet}\sl
    a) The inheritance structure of the BJet ensures common interfaces with
    the other AOD objects. The BJet has a vector of IBInfo objects which
    contain the tagging information from the various tagging tools. b) The 
    information from each tag is stored in a BInfo-object. All 
    BInfo-objects inherit from the interface base-class IBInfo.}
\end{figure}

\subsection{The {\tt BInfo}-Object}\label{sec:binfo}
Every tag has its distinctive BInfo object where all necessary tagging 
information is stored. To enable the combination of the tags all 
BInfo-objects inherit from a common interface base-class {\tt IBInfo}
(fig.~\ref{fig:bjet}b). The {\tt IBInfo}-object forces the implementation 
of the signal- and background-likelihood needed for a correctly normalised 
statistical combination of the tags in the main algorithm.  

%\begin{figure}[htbp]
%  \centerline{\epsfig{file=images/BInfoBasic.eps,width=5.0cm}}
%  \caption[]{\label{fig:binfo}\sl
%    The information from each tag is stored in a BInfo-object. All 
%    BInfo-objects inherit from the interface base-class IBInfo.}
%\end{figure}



\section{Implementation}\label{implementation}

\subsection{Implemented B-Tagging Tools}

\begin{figure}[htbp]
  \centerline{\epsfig{file=images/BTag.eps,width=12.0cm}}
  \caption[]{\label{fig:recvertex}\sl
    The scematic implementation of the reconstructed vertex class
    {\tt RecVertex}.}
\end{figure}







% -------------------------------------------------------------------------

\subsection{Lifetime Tag}
The lifetime tag uses the larger impact significances of tracks comming 
from decaying b-flavoured particles. Due to the on average longer lifetime
of b-mesons and b-baryons the decay will take place away from the primary
vertex and this resulting in larger impact significances. The impact 
significance $s$ is defined as
\begin{equation}
  s=\frac{d}{\sigma_d},
\end{equation}
where $d$ can be the distance in $z$ (1D), in the $r\phi$-plane (2D)
or the three-dimensional distance in $z-r\phi$ (3D) of the point of
closest approach of the track to the reconstructed primary vertex. The
sign of $s$ is determined by the crossing point of the track with
respect to the jet-axis. If the track crosses ahead(behind) of the
primary vertex $s$ gets a positive(negative) sign (see
Fig.~\ref{fig:ipsign})
.
\begin{table}[h]
  \begin{center}\caption[]{\label{tab:lumi_st}\sl 
    Tracks passing the below quality cuts are used in the signed
    impact parameter tag. $d_0$ and $z_0$ are given with respect to
    the reconstructed primary vertex.\\}
    \begin{tabular}{|l|l|} \hline
      Cut   & Value \\ \hline
      $p_t$ & $\ge$ 1~\gev \\
      $d_0$ & $\le$ 3.5~mm \\
      $z_0$ & $\le$ 5~mm \\
      Number of Pixel+SCT Hits & $\ge$ 9 \\
      Number of Pixel Hits & $\ge$ 2 \\
      Number of B-Layer Hits & $\ge$ 1 \\ \hline
    \end{tabular}
  \end{center}
\end{table}

\begin{figure}[htbp]
  \centerline
  {
    \epsfig{file=images/SignedImpactSignificance1D.eps,width=5.5cm}
    \epsfig{file=images/SignedImpactSignificance2D.eps,width=5.5cm}
    \epsfig{file=images/SignedImpactSignificance3D.eps,width=5.5cm}
  }
  \caption[]{\label{fig:recvertex}\sl The distributions of the signed
    impact parameter determined from (a) the $z$-distance (1D), (b)
    the $r\phi$-plane distance (2D), and (c) from the $z-r\phi$
    distance (3D) of the point of closest approach of the track to the
    reconstructed primary vertex.}
\end{figure}

\subsection{Secondary Vertex Fit}
In addition to the higher values of the impact significance it is
possible to reconstruct secondary vertices in the b-flavoured jets.
There are several ways of seeding the secondary verices and assigning
tracks to the vertices. One way is to fit vertices with all pairs of
tracks, retaining the fit with the highest probability and fitting the
remaining tracks of the jet to this vertex. All tracks below a certain
fit probability will be rejected. This approach is implemented in the
build-up (BU) algorithm. An other approach is to start with all tracks
in the jet, fit a vertex, and reject tracks below a certain fixed fit
probability. This has been implemented in the tear-down (TD)
algorithm.  

After the secondary vertex finding properties of the vertex, such as
mass, fit probability, multiplicity and distance from the
reconstructed primary vertex can be studied. In Fig.~\ref{fig:BUvars}
and Fig.~\ref{fig:TDvars} the properties of the vertex found by the
build-up and the tear-down method respectively is shown.

\begin{figure}[htbp]
  \centerline
  {
    \epsfig{file=images/BUSecVtxMass.eps,width=5.5cm}
    \epsfig{file=images/BUSecVtxMultiplicity.eps,width=5.5cm}
    \epsfig{file=images/BUSecVtxProb.eps,width=5.5cm}
  }  
  \centerline
  {
    \epsfig{file=images/BU2DSecVtxDistance.eps,width=5.5cm}
    \epsfig{file=images/BU3DSecVtxDistance.eps,width=5.5cm}
  }
  \caption[]{\label{fig:BUvars}\sl
    Discriminating variables from the build-up secondary vertex tag.  }
\end{figure}

\begin{figure}[htbp]
  \centerline
  {
    \epsfig{file=images/TDSecVtxMass.eps,width=5.5cm}
    \epsfig{file=images/TDSecVtxMultiplicity.eps,width=5.5cm}
    \epsfig{file=images/TDSecVtxProb.eps,width=5.5cm}
  }  
  \centerline
  {
    \epsfig{file=images/TD2DSecVtxDistance.eps,width=5.5cm}
    \epsfig{file=images/TD3DSecVtxDistance.eps,width=5.5cm}
  }
  \caption[]{\label{fig:TDvars}\sl
    Discriminating variables from the tear-down secondary vertex tag.  }
\end{figure}


The resulting likelihood, combining the vertex mass, multiplicity, 
fit probability and the two-dimensional distance can be seen in 
Fig.~\ref{fig:secvtxlh}.

\begin{figure}[htbp]
  \centerline
  {
    \epsfig{file=images/BULikelihood.eps,width=5.5cm}
    \epsfig{file=images/TDLikelihood.eps,width=5.5cm}
  }
  \caption[]{\label{fig:TDvars}\sl
    Discriminating variables from the tear-down secondary vertex tag.  }
\end{figure}


\section{Combined b-Tagging Likelihood}
In the default configuration the b-Tagger combines the following 
tags in one likelihood to deliver the combined b-tag of a jet:
\begin{itemize}
\item 1D Signed impact significance
\item 2D Signed impact significance
\item Build-up secondary vertex fit
\end{itemize}
The distribution of the combined likelihood is shown in Fig.~\ref{fig:comblh}(a)
and the efficiency for bottom- and light-jets is shown in Fig.~\ref{fig:comblh}(b). 

\begin{figure}[htbp]
  \centerline
  {
    \epsfig{file=images/CombinedLikelihood.eps,width=7.0cm}
    \epsfig{file=images/CombinedEfficiency.eps,width=7.0cm}
  }
  \caption[]{\label{fig:comblh}\sl (a) The distribution of the
    combined likelihood and (b) the efficiency for bottom- and
    light-jets is shown in Fig.~\ref{fig:comblh}(b).}
\end{figure}




% -------------------------------------------------------------------------

\begin{appendix}

\section{How to Run the B-Tagger}
This section describes how to use the b-tagger as a user within the 
athena framework.

The default version of the release can be used by including the following line 
\begin{verbatim}
  include( "BTaggingAlgs/BTagging_jobOptions.py" ); 
\end{verbatim}
into your Python job-options file. Please note that the b-tagger must
run after the tracking, the primary vertex finding and the {\tt TrackParticle}
creation. Currently external jet-finding is not needed for the b-tagging
since the jet-finding is done on-the-fly when running the algorithm.
This, however, can be changed to use any other jet-finding algorithm you
may desire.   

The output level of the b-tagger can be set with
\begin{verbatim}
  BJetBuilder.OutputLevel = DEBUG;
\end{verbatim}

\subsection{"Jet input flow" for the BTagger}
The Btagger can be run with different kind of jets.
\begin{enumerate}
\item KtTowerJets based on Cells: since the BTagger needs tracks to
perform the tags, TrackParticles have to be associated to the found
Jets. This is automatically done by the TrackParticleAndJetMerger
algorithm.
\item KtJets based on TrackParticles (an instance of the JetAlgorithm
will be run by the BTagger with TrackParticles as input)
\item Jets from either (1) or (2) can be sent through a TruthMatching
algorithm which either only take out "true" BJets (if in SIG modus,
see Appendix about flags) or only take light jets (when in BKG
modus). By default the TruthMatching algorithm is turned of (NONE
flag).
\end{enumerate} 

The possible Jet input is illustrated in Figure XX.
\begin{figure}[htbp]
 \centerline{\epsfig{file=images/BTaggin_jetSourceFlow.eps}}
 \caption[]{\label{fig:jetinput}\sl Possible Jet input collections for
    the BTagger. It can either take Jets based on TrackParticles or
    Cells (with TrackParticles associated to Jets). In addition a
    truth matching can be done before the JetCollection is given to
    the b-tagging algorithm. The flags used to steer the input are
    written in the diagram. If applied they have to be suffixed with
    \textit{BTaggingAlgsFlags.}. }
\end{figure}





\subsection{General steering of the BTagger}
From within your jobOptions file you can set general steering
variables which control the execution of the BTagger. These global
python flags enable the user to control the BTagging without the
necessity of going into its jobOptions files. All flags have default
values so that the BTagger can also be run without any modifications.

If you want to use the global steering flags you have to import them
in your jobOptions file with:
\begin{verbatim}
if not 'BTaggingAlgsFlags' in dir():
  from BTaggingAlgs.BTaggingAlgsFlags import BTaggingAlgsFlags
\end{verbatim}
If you want to change a flag from its default value you have to do so
\textbf{after} the import lines and \textbf{before} including the
general BTagging jobOptions file. E.g. for changing the modus the
BTagger should run in from \textit{analysis} (which is default) to
\textit{reference} you just need to add this line:
\begin{verbatim}
BTaggingAlgsFlags.modus = 'reference'
\end{verbatim}
The next tables give a detailed description of the flags, their
default values and all possible values they can be set to (remember,
to change a flag you always need to add \textit{BTaggingAlgsFlags.} in
front of it).

\begin{tabular}{l|l|l|l}
Flag & default & others & description \\ \hline modus & 'analysis' &
'reference' & choose the mode the BTagger should run in \\
jetFinderBasedOn & 'TrackParticle' & 'Cells" & Jets should be found on
the basis of TrackParticles or Cells \\ truthMatching & 'NONE' &
'SIG', 'BKG' & the algorithm can do a truth matching to be 'sure' that
it was a bjet or a light jet \\ \hline trackerUsedForRefHistos &
'xKal' & 'iPat' & choose the tracker with which the ref samples were
made. Right now only xKalmann is supported.\\ sampleUsedForRefHistos &
'Hxx' & 'Zxx' & choose which sample was used for the reference
histos. Default is Hbb for signal and Huu for background.\\
%\caption[]{\label{table:generalflags}\sl
%Flags to steer the general behaviour of the BTagger.}
\end{tabular} 

\begin{tabular}{l|l|l}
  Flag & default & description\\ \hline
  doBTagCBNT & OFF & to run the BTagging CBNT algorithms.\\
  doBTagJetCBNT & OFF & to run the CBNT of the JetAlgorithm.\\
\end{tabular} 

\begin{tabular}{l|c|l}
Flag & default & description\\ \hline
lifetime1D & OFF & lifetime tag based on $z_0$\\
lifetime2D & ON & lifetime tag based on $d_0$ ($R-\phi$ plane)\\
lifetime3D & OFF & lifetime tag based on the 3-dimensional impact parameter $a_0$\\
secVtxFitBU & ON & secondary vertex tag based on a build-up vertex finder strategy\\
secVtxFitTD & OFF & secondary vertex tag based on a tear-down vertex finder strategy\\
\end{tabular} 

\begin{tabular}{l|c|l}
Flag & default & description\\
\hline
useLifetime1DForLH & ON & include lifetime1D info in the combined LH calculation\\
useLifetime2DForLH & ON & include lifetime2D info in the combined LH calculation\\
useLifetime3DForLH & ON & include lifetime3D info in the combined LH calculation\\
useSecVtxFitBUForLH & ON & include secVtxFitBU info in the combined LH calculation\\
useSecVtxFitTDForLH & ON & include secVtxFitTD info in the combined LH calculation\\
\end{tabular} 

\subsection{How to Control the Used Tags and Variables}
If you want to select specific tags to run or variables to use within
the tags you can comfortably control this with job-options.
To select a specific tag, in this example the {\tt BJetLifetimeTag}, you 
add the tagging tool name and an instance name:
\begin{verbatim}
  BJetBuilder.useTools += [ "BJetLifetimeTag" ];
  BJetBuilder.toolInstanceNames += [ "myBJetLifetimeTagTool1D" ];
\end{verbatim}
That will cause the {\tt BJetBuilder} to run the {\tt BJetLifetimeTag} with
the specified instance name "myBJetLifetimeTagTool1D". For this instance you
may want to specify certain cuts and variables. This is done with the 
job-options:
\begin{verbatim}
  Z0LifetimeTagTool = Service ("ToolSvc.myBJetLifetimeTagTool1D")
  Z0LifetimeTagTool.LifetimeModus = "1D"
  Z0LifetimeTagTool.useVariables = [ "significance1D" ]
  Z0LifetimeTagTool.minpt = 1.*GeV           # min pt of track
  Z0LifetimeTagTool.maxd0wrtPrimVtx = 3.5*mm # wrt to prim vertex
  Z0LifetimeTagTool.maxz0wrtPrimVtx = 5.*mm  # wrt to prim vertex
\end{verbatim}
This tells the tool to use the one-dimensional signed impact significance
(in $z$-direction) {\tt "significance1D"} and to apply the stated quality cuts on the 
tracks. The same tool can be run multiple times with different instance names:
\begin{verbatim}
  BJetBuilder.useTools += [ "BJetLifetimeTag" ];
  BJetBuilder.toolInstanceNames += [ "myBJetLifetimeTagTool2D" ];
\end{verbatim}
and specific job-options for this tag:
\begin{verbatim}
  D0LifetimeTagTool = Service ("ToolSvc.myBJetLifetimeTagTool2D")
  D0LifetimeTagTool.LifetimeModus = "2D"
  D0LifetimeTagTool.useVariables = [ "significance2D" ]
  D0LifetimeTagTool.minpt = 2.*GeV           # min pt of track
  D0LifetimeTagTool.maxd0wrtPrimVtx = 2.5*mm # wrt to prim vertex
  D0LifetimeTagTool.maxz0wrtPrimVtx = 3.*mm  # wrt to prim vertex
\end{verbatim}
This tells the tool to run the instance using the two-dimensional signed
impact significance ($r-\phi$-plane) as well. All specified and executed 
tags are then combined (with a one-dimensional likelihood method so far) 
in the {\tt BJetBuilder}. 

\section{How to Add Your Favourite Tag}

\end{appendix}

\input literature.tex
\end{document}

